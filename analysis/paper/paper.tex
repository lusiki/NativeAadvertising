\documentclass[preprint, 3p,
authoryear]{elsarticle} %review=doublespace preprint=single 5p=2 column
%%% Begin My package additions %%%%%%%%%%%%%%%%%%%

\usepackage[hyphens]{url}

  \journal{An awesome journal} % Sets Journal name

\usepackage{graphicx}
%%%%%%%%%%%%%%%% end my additions to header

\usepackage[T1]{fontenc}
\usepackage{lmodern}
\usepackage{amssymb,amsmath}
% TODO: Currently lineno needs to be loaded after amsmath because of conflict
% https://github.com/latex-lineno/lineno/issues/5
\usepackage{lineno} % add
\usepackage{ifxetex,ifluatex}
\usepackage{fixltx2e} % provides \textsubscript
% use upquote if available, for straight quotes in verbatim environments
\IfFileExists{upquote.sty}{\usepackage{upquote}}{}
\ifnum 0\ifxetex 1\fi\ifluatex 1\fi=0 % if pdftex
  \usepackage[utf8]{inputenc}
\else % if luatex or xelatex
  \usepackage{fontspec}
  \ifxetex
    \usepackage{xltxtra,xunicode}
  \fi
  \defaultfontfeatures{Mapping=tex-text,Scale=MatchLowercase}
  \newcommand{\euro}{€}
\fi
% use microtype if available
\IfFileExists{microtype.sty}{\usepackage{microtype}}{}
\usepackage[]{natbib}
\bibliographystyle{elsarticle-harv}

\ifxetex
  \usepackage[setpagesize=false, % page size defined by xetex
              unicode=false, % unicode breaks when used with xetex
              xetex]{hyperref}
\else
  \usepackage[unicode=true]{hyperref}
\fi
\hypersetup{breaklinks=true,
            bookmarks=true,
            pdfauthor={},
            pdftitle={Native advertising as a content marketing strategy},
            colorlinks=false,
            urlcolor=blue,
            linkcolor=magenta,
            pdfborder={0 0 0}}

\setcounter{secnumdepth}{5}
% Pandoc toggle for numbering sections (defaults to be off)

% Pandoc syntax highlighting
\usepackage{color}
\usepackage{fancyvrb}
\newcommand{\VerbBar}{|}
\newcommand{\VERB}{\Verb[commandchars=\\\{\}]}
\DefineVerbatimEnvironment{Highlighting}{Verbatim}{commandchars=\\\{\}}
% Add ',fontsize=\small' for more characters per line
\usepackage{framed}
\definecolor{shadecolor}{RGB}{248,248,248}
\newenvironment{Shaded}{\begin{snugshade}}{\end{snugshade}}
\newcommand{\AlertTok}[1]{\textcolor[rgb]{0.94,0.16,0.16}{#1}}
\newcommand{\AnnotationTok}[1]{\textcolor[rgb]{0.56,0.35,0.01}{\textbf{\textit{#1}}}}
\newcommand{\AttributeTok}[1]{\textcolor[rgb]{0.13,0.29,0.53}{#1}}
\newcommand{\BaseNTok}[1]{\textcolor[rgb]{0.00,0.00,0.81}{#1}}
\newcommand{\BuiltInTok}[1]{#1}
\newcommand{\CharTok}[1]{\textcolor[rgb]{0.31,0.60,0.02}{#1}}
\newcommand{\CommentTok}[1]{\textcolor[rgb]{0.56,0.35,0.01}{\textit{#1}}}
\newcommand{\CommentVarTok}[1]{\textcolor[rgb]{0.56,0.35,0.01}{\textbf{\textit{#1}}}}
\newcommand{\ConstantTok}[1]{\textcolor[rgb]{0.56,0.35,0.01}{#1}}
\newcommand{\ControlFlowTok}[1]{\textcolor[rgb]{0.13,0.29,0.53}{\textbf{#1}}}
\newcommand{\DataTypeTok}[1]{\textcolor[rgb]{0.13,0.29,0.53}{#1}}
\newcommand{\DecValTok}[1]{\textcolor[rgb]{0.00,0.00,0.81}{#1}}
\newcommand{\DocumentationTok}[1]{\textcolor[rgb]{0.56,0.35,0.01}{\textbf{\textit{#1}}}}
\newcommand{\ErrorTok}[1]{\textcolor[rgb]{0.64,0.00,0.00}{\textbf{#1}}}
\newcommand{\ExtensionTok}[1]{#1}
\newcommand{\FloatTok}[1]{\textcolor[rgb]{0.00,0.00,0.81}{#1}}
\newcommand{\FunctionTok}[1]{\textcolor[rgb]{0.13,0.29,0.53}{\textbf{#1}}}
\newcommand{\ImportTok}[1]{#1}
\newcommand{\InformationTok}[1]{\textcolor[rgb]{0.56,0.35,0.01}{\textbf{\textit{#1}}}}
\newcommand{\KeywordTok}[1]{\textcolor[rgb]{0.13,0.29,0.53}{\textbf{#1}}}
\newcommand{\NormalTok}[1]{#1}
\newcommand{\OperatorTok}[1]{\textcolor[rgb]{0.81,0.36,0.00}{\textbf{#1}}}
\newcommand{\OtherTok}[1]{\textcolor[rgb]{0.56,0.35,0.01}{#1}}
\newcommand{\PreprocessorTok}[1]{\textcolor[rgb]{0.56,0.35,0.01}{\textit{#1}}}
\newcommand{\RegionMarkerTok}[1]{#1}
\newcommand{\SpecialCharTok}[1]{\textcolor[rgb]{0.81,0.36,0.00}{\textbf{#1}}}
\newcommand{\SpecialStringTok}[1]{\textcolor[rgb]{0.31,0.60,0.02}{#1}}
\newcommand{\StringTok}[1]{\textcolor[rgb]{0.31,0.60,0.02}{#1}}
\newcommand{\VariableTok}[1]{\textcolor[rgb]{0.00,0.00,0.00}{#1}}
\newcommand{\VerbatimStringTok}[1]{\textcolor[rgb]{0.31,0.60,0.02}{#1}}
\newcommand{\WarningTok}[1]{\textcolor[rgb]{0.56,0.35,0.01}{\textbf{\textit{#1}}}}

% tightlist command for lists without linebreak
\providecommand{\tightlist}{%
  \setlength{\itemsep}{0pt}\setlength{\parskip}{0pt}}







\begin{document}


\begin{frontmatter}

  \title{Native advertising as a content marketing strategy}
    \author[Croatian Catholic University]{Davor Trbusic%
  \corref{cor1}%
  \fnref{1}}
   \ead{davor.trbusic@unicath.hr} 
    \author[Faculty of Croatian Studies]{Drazen Males%
  %
  }
   \ead{dmales@fhs.hr} 
    \author[Croatian Catholic University]{Luka Sikic%
  %
  \fnref{2}}
   \ead{luka.sikic@unicath.hr} 
      \affiliation[Croatian Catholic University]{
    organization={Croatian Catholic University},addressline={Ilica
242},city={Zagreb},postcode={10000},country={Croatia},}
    \affiliation[Faculty of Croatian Studies]{
    organization={Faculty of Croatian Studies},addressline={Borongajska
cesta 83d},city={Zagreb},postcode={10000},country={Croatia},}
    \cortext[cor1]{Corresponding author}
    \fntext[1]{This is the corresponding author.}
    \fntext[2]{Co-author of the study.}
  
  \begin{abstract}
  This research focuses on analyzing the content elements of native
  advertisements on the most popular Croatian online portals, with the
  aim of identifying key characteristics of headlines and content, as
  well as their correlation with audience reach. The study was conducted
  on a sample of 543 native ads published on six leading Croatian online
  portals from December 2021 to May 2022. The content analysis method
  was used, focusing on two general categories: headline characteristics
  and content characteristics. The analysis found that nearly 80\% of
  the headlines had at least one clickbait characteristic, with the most
  commonly used styles including uncertainty (44.38\%), use of numbers
  (9.39\%), and emphasis on emotions (5.89\%). The headlines were
  predominantly declarative sentences, with one-fifth being
  interrogative sentences, while the connection between the ad headline
  and the advertiser's brand name was very rarely present. In terms of
  content, visual elements were present in all ads, with photographs
  being the dominant element, and visual identity elements of the
  advertiser, such as logos, were included in 34.44\% of the ads.
  Regarding sources in native ads, 47.33\% of the ads did not use
  statements, while the most common sources were individuals from the
  organization (19.71\%). Statements from direct users of the products
  or services were present in 6.81\% of the cases, while statements from
  brand ambassadors and influencers were present in smaller percentages
  (4.79\% and 3.5\%, respectively). These findings highlight the
  dominance of clickbait headlines and the importance of visual identity
  in native advertising, while direct brand association and the use of
  various sources, including direct users, appear less frequently. A
  comparison of different types of headlines and sources with ad reach
  indicates specific practices in native advertising across different
  industries.
  \end{abstract}
    \begin{keyword}
    native advertising \sep content analysis \sep 
    headline characteristics
  \end{keyword}
  
 \end{frontmatter}

\hypertarget{introduction}{%
\section{Introduction}\label{introduction}}

The concentration of so-called hybrid forms of media content in the
contemporary online environment is becoming increasingly intense
(Balasubramanian, 1994; Macnamara, 2014; Taiminen, Luoma-aho \&
Tolvanen, 2015). Among the most prominent forms are sponsored content
(Tutaj \& Reijmersdal, 2012), content marketing (Pulizzi, 2014; Pulizzi
\& Piper, 2023), brand journalism (Cole \& Greer, 2013), and native
advertising (Verčič \& Tkalac Verčič, 2016). For most researchers, the
latter term will serve as an umbrella term that attempts to simplify the
classification of hybrid forms, as native advertising by definition
implies a paid advertisement that resembles editorial content, aiming to
attract the target audience with useful content while simultaneously
promoting the brand, values, and products (Cole \& Greer, 2013; Howe \&
Teufel, 2014; Wojdynski \& Evans, 2016). On the other hand, the research
by Taiminen et al.~(2015) showed that most public relations and
marketing professionals preferred to use the term ``content marketing''
when discussing commercial hybrid forms of online content. In addition
to the obvious terminological confusion, three other problematic areas
of native advertising need to be highlighted:

Application area: Is native advertising a public relations technique, a
marketing tool, or a tactic that does not differ from traditional
advertising in any way?

Authorship of native ads: Who creatively signs them and who is
responsible for their content? Legal and ethical framework: How is
native advertising legally defined, and what is the practice of media
houses in labeling such advertisements (Trbušić, Maleš \& Šikić, 2024).

The theoretical basis of this paper is the thesis that native
advertising is ``an extension of content marketing and provides
marketers with a platform to place content in front of a wider audience
they wouldn't ordinarily reach'' (Crook, 2022). Although there are
authors who highlight certain criteria by which content marketing
differs from native advertising, which will be described in more detail
later in the paper, the ultimate goal is common: to drive profitable
customer action. The profitability of such market communication for
organizations and brands is demonstrated by recent metrics. Figures show
that 47\% of marketing professionals globally believe in the
effectiveness of this type of advertising (Kloot, 2022), and that in
2023, spending on native advertising in the United States increased by
12\% compared to the previous year (eMarketer, 2023). Thus, with more
than 97 billion US dollars invested, native advertising holds the
largest share (59.7\%) of the total expenditure for all types of
advertising (eMarketer, 2023).

\begin{Shaded}
\begin{Highlighting}[]
\FunctionTok{library}\NormalTok{(dplyr)}
\end{Highlighting}
\end{Shaded}

\begin{verbatim}
## Warning: package 'dplyr' was built under R version 4.3.3
\end{verbatim}

\begin{verbatim}
## 
## Attaching package: 'dplyr'
\end{verbatim}

\begin{verbatim}
## The following objects are masked from 'package:stats':
## 
##     filter, lag
\end{verbatim}

\begin{verbatim}
## The following objects are masked from 'package:base':
## 
##     intersect, setdiff, setequal, union
\end{verbatim}

\begin{Shaded}
\begin{Highlighting}[]
\FunctionTok{library}\NormalTok{(readr)}
\FunctionTok{library}\NormalTok{(ggplot2)}
\end{Highlighting}
\end{Shaded}

\begin{verbatim}
## Warning: package 'ggplot2' was built under R version 4.3.3
\end{verbatim}

\begin{Shaded}
\begin{Highlighting}[]
\FunctionTok{library}\NormalTok{(ggthemes)}
\end{Highlighting}
\end{Shaded}

\begin{verbatim}
## Warning: package 'ggthemes' was built under R version 4.3.3
\end{verbatim}

\begin{Shaded}
\begin{Highlighting}[]
\FunctionTok{library}\NormalTok{(tidyverse)}
\end{Highlighting}
\end{Shaded}

\begin{verbatim}
## -- Attaching core tidyverse packages ------------------------ tidyverse 2.0.0 --
## v forcats   1.0.0     v stringr   1.5.1
## v lubridate 1.9.3     v tibble    3.2.1
## v purrr     1.0.2     v tidyr     1.3.0
\end{verbatim}

\begin{verbatim}
## -- Conflicts ------------------------------------------ tidyverse_conflicts() --
## x dplyr::filter() masks stats::filter()
## x dplyr::lag()    masks stats::lag()
## i Use the conflicted package (<http://conflicted.r-lib.org/>) to force all conflicts to become errors
\end{verbatim}

\begin{Shaded}
\begin{Highlighting}[]
\FunctionTok{library}\NormalTok{(readxl)}
\FunctionTok{library}\NormalTok{(xlsx)}
\FunctionTok{library}\NormalTok{(here)}
\end{Highlighting}
\end{Shaded}

\begin{verbatim}
## Warning: package 'here' was built under R version 4.3.3
\end{verbatim}

\begin{verbatim}
## here() starts at C:/Users/lukas/Dropbox/Članci/NativeAdvertising
\end{verbatim}

\hypertarget{native-advertising-and-content-marketing-two-faces-of-the-same-coin}{%
\section{Native advertising and content marketing -- two faces of the
same
coin?}\label{native-advertising-and-content-marketing-two-faces-of-the-same-coin}}

Charmaine Du Plessis (2015a) defines content marketing as sharing brand
content via owned media to gain earned media, which clearly implies that
it is not a type of media content that is paid for, as (native)
advertising certainly is. In this context, Du Plessis employs the
classic PESO model of media content in public relations (Dietrich,
2014), which distinguishes paid media (advertising or other sponsored
content); earned media (publicity generated from media pitches or news
releases); shared media (content shared and re-shared on social media);
and owned media (such as company's website, newsletter, or email
communications). The order in the PESO model has long reflected the
priorities of organizations in media strategy planning. However, recent
research (Macnamara, Lwin, Adi \& Zerfass, 2016) indicates a shift in
the order in practice, with the dominance shifting to the SOEP model
instead of the classic PESO model. The reasons for this are manifold,
but the most indicative is that owned and earned media content are
trusted more than paid media, while shared media (i.e.,
``recommendations from people I know'' and ``consumer opinions posted
online'') are the most trusted information sources of all (Nielsen,
2015).

Like Du Plessis, other authors (Handley \& Chapman, 2011; Halvorson \&
Rach, 2012) also differentiate content marketing from other forms of
hybrid content exclusively based on the distribution channel (owned
media). However, there are authors who leave much more room for
interpretation, not strictly confining it to owned media that later
generates earned media (publicity). Their definitions are more focused
on the strategy, intent, and ultimate goal of content marketing, rather
than on the tactics and type of media content within the PESO framework.
For instance, they highlight that content marketing is ``a strategic
marketing approach focused on creating and distributing valuable,
relevant, and consistent content'' (Pulizzi \& Piper, 2023; CMI, 2024),
which is disseminated on various platforms such as newspapers,
magazines, TV, or radio channels (Holliman \& Rowley, 2014).

The distribution of valuable, relevant, and consistent content is also
at the core of native advertising (Campbell \& Marks, 2015; Kim, 2017),
which is especially evident in so-called branded or native content.
According to the Interactive Advertising Bureau (IAB), there are three
fundamental forms of native advertising: In-Feed/In-Content native
advertising, Content Recommendation Ads, and Branded/native content. The
latter refers to ``paid content by a brand that is published in the same
format as full-fledged editorial content on the publisher's website,
usually in collaboration with the publisher's own teams'' (IAB, 2019,
p.~14). This form of native advertising will be the focus of the
research part of this paper for two main reasons: it conceptually
correlates with the forms and tactics of content marketing, and it is
the only form among the mentioned types of native advertising that can
be approached analytically.

Types and tactics of content marketing typically include blogs,
infographics, video content, photos, webinars, podcasts and social media
(Halligan \& Shah, 2010; Handley \& Chapman, 2011; Bloomstein, 2012;
Wuebben, 2012; Vibrant Publishers, 2020), but also paid ads (Baker,
2024), such as sponsored advertising options on social media, affiliate
marketing, remarketing, native ads etc. (Jefferson \& Tanton, 2015). It
is yet another proof of the inconsistent interpretation of content
marketing as something that is disseminated exclusively via owned media.
It is more appropriate to consider content marketing as a set of tactics
that, among other things, can drive traffic back to the website of some
company or product (owned media). It is also noticeable that some
authors of definitions avoid identifying content marketing with product
advertising which aims at the purchase of a product or service (Patrutiu
Baltes, 2015), seeking to further differentiate it from other related
marketing tactics. The fact is that content marketing is part of an
inbound marketing strategy, which, unlike the outbound strategy, is
focused on ``attracting valuable consumers (potential, existing, or
aspirational) who choose to interact with a particular company that
provides them with something useful'' (Opreana \& Vinerean, 2015,
p.~29). However, it is misguided to claim that the ultimate goal of any
marketing communication, including content marketing, is not sales or
gaining profit (Rowley, 2008; Pulizzi \& Piper, 2023). Furthermore, both
academics and practitioners agree that content marketing relies on a
consumer-centric strategy (Ho, Pang \& Choy, 2020). In conclusion,
regardless of the type of marketing communication, whether it is content
marketing, native advertising, or brand journalism - although one can
discuss nuanced tactics and approaches to attract consumers into the
marketing funnel and later motivate them for retention and advocacy
phases - profitable customer action remains the intrinsic motive of
every organization or brand.

Previous research does not encompass the analysis of native advertising
as part of content marketing; rather, these concepts have been the focus
of separate studies. Thus, insights in the context of native advertising
relate to the analysis of readers' cognitive perception of native ads
and their persuasive elements (Wojdynski \& Evans, 2016), analysis of
communication strategies in native ads (Wang \& Li, 2017), perception of
the organization or the sponsor of native ads and its socially
responsible activities (Jung \& Heo, 2018; Krouwer, Poels \& Paulussen,
2019; Beckert, Koch, Viererbl, Denner, \& Peter, 2020; Wu \& Overton,
2021), perception of the credibility of online portals that publish
native ads (Howe \& Teufel, 2014), and analysis of the basic features of
native ads based on the most-read selected Croatian online portals
(Trbušić et al., 2024). Research in the field of content marketing
primarily reflects valuable theoretical contributions, such as
conceptual definitions, classification of types and forms of content
marketing, identification of strategies and tactics of content marketing
in the digital environment (Koiso-Kanttila, 2004; Rowley, 2008; Handley
\& Chapman, 2011; Bloomstein, 2012; Wuebben, 2012; Pulizzi, 2014;
Jefferson \& Tanton, 2015; Vinerean, 2017; Pulizzi \& Piper, 2023), and
the definition of six fundamental elements of content marketing which
should be considered by marketers (Du Plessis, 2015a; Du Plessis,
2015b). Empirically, content marketing has been analyzed as part of B2B
strategies on digital channels (Holliman \& Rowley, 2014) or in the
context of content creation practices in successful companies (Du
Plessis, 2015c; Ho et al., 2020). Additionally, studies have addressed
the causal linkage between content marketing and online consumer
behavior (Du Plessis, 2022) and content marketing as part of business
strategy in the luxury industry (Rios, 2016).

\hypertarget{methodology-and-analytical-design}{%
\section{Methodology and analytical
design}\label{methodology-and-analytical-design}}

Cilj istraživanja je utvrditi osnovne karakteristike sadržajnih
elemenata nativnih oglasa (kao jedne od strategija sadržajnog
marketinga) na promatranim najčitanijim hrvatskim internetskim
portalima. Specifični ciljevi istraživanja usmjereni su na promatranje
dviju širih cjelina. Riječ je o značajkama naslova (npr. vrste rečenica
u naslovima, zastupljenost i vrste clikbait naslova, navođenje naziva
brenda) te o analizi elemenata unutar samog sadržaja nativnih oglasa
(npr. navođenje izvora tj. upotrebljavanje izjava sugovornika,
korištenje opreme i vizualnog identiteta). Spomenute kategorije nisu
izabrane proizvoljno, već su definirane sukladno vrstama i tipologiji
sadržajnog marketinga (see p.~5). Osim toga, cilj je utvrditi postoji li
značajna korelacija između karakteristikâ naslova i sadržaja nativnih
oglasa s dosegom publike. Jednako tako, pozornost je u radu usmjerena i
na specifičnosti koje se u ovom vidu oglašavanja primjećuju kod
različitih industrija, tj. različitih praksi kojima pribjegavaju
naručitelji nativnih oglasa.

\begin{Shaded}
\begin{Highlighting}[]
\NormalTok{original }\OtherTok{\textless{}{-}} \FunctionTok{read.xlsx}\NormalTok{(}\FunctionTok{here}\NormalTok{(}\StringTok{"analysis"}\NormalTok{, }\StringTok{"data"}\NormalTok{, }\StringTok{"raw\_data"}\NormalTok{, }\StringTok{"native\_articles.xlsx"}\NormalTok{), }\AttributeTok{sheetIndex =} \DecValTok{1}\NormalTok{) }\SpecialCharTok{\%\textgreater{}\%} \FunctionTok{mutate}\NormalTok{(}\AttributeTok{V1 =} \FunctionTok{as.numeric}\NormalTok{(V1))}
\NormalTok{variables }\OtherTok{\textless{}{-}} \FunctionTok{read.xlsx}\NormalTok{(}\FunctionTok{here}\NormalTok{(}\StringTok{"analysis"}\NormalTok{, }\StringTok{"data"}\NormalTok{, }\StringTok{"raw\_data"}\NormalTok{, }\StringTok{"native\_research.xlsx"}\NormalTok{), }\AttributeTok{sheetIndex =} \DecValTok{1}\NormalTok{)}

\NormalTok{dta }\OtherTok{\textless{}{-}} \FunctionTok{merge}\NormalTok{(}
\NormalTok{  original }\SpecialCharTok{\%\textgreater{}\%} \FunctionTok{filter}\NormalTok{(V1 }\SpecialCharTok{\%in\%}\NormalTok{ variables}\SpecialCharTok{$}\NormalTok{V1),}
\NormalTok{  variables }\SpecialCharTok{\%\textgreater{}\%} \FunctionTok{select}\NormalTok{(}\SpecialCharTok{{-}}\FunctionTok{c}\NormalTok{(}\DecValTok{23}\NormalTok{, }\DecValTok{24}\NormalTok{, }\DecValTok{25}\NormalTok{)),}
  \AttributeTok{by =} \StringTok{"V1"}\NormalTok{,}
  \AttributeTok{all.x =} \ConstantTok{TRUE}
\NormalTok{)}

\CommentTok{\# Convert the \textquotesingle{}DATE\textquotesingle{} column to Date format}
\NormalTok{dta}\SpecialCharTok{$}\NormalTok{DATE }\OtherTok{\textless{}{-}} \FunctionTok{as.Date}\NormalTok{(dta}\SpecialCharTok{$}\NormalTok{DATE)}









\NormalTok{stemmed }\OtherTok{\textless{}{-}} \FunctionTok{readRDS}\NormalTok{(}\StringTok{"C:/Users/Lukas/Dropbox/Mediatoolkit/native\_token\_stemm.rds"}\NormalTok{)}
\NormalTok{stemmed }\OtherTok{\textless{}{-}} \FunctionTok{readRDS}\NormalTok{(}\StringTok{"C:/Users/lukas/Dropbox/Članci/Native rad/native\_token\_stemm.rds"}\NormalTok{)}

\NormalTok{stemmed }\OtherTok{\textless{}{-}}\NormalTok{ stemmed }\SpecialCharTok{\%\textgreater{}\%}
  \FunctionTok{rename}\NormalTok{(}\StringTok{"wordp"} \OtherTok{=} \StringTok{"word"}\NormalTok{) }\SpecialCharTok{\%\textgreater{}\%}
  \FunctionTok{rename}\NormalTok{(}\StringTok{"word"} \OtherTok{=} \StringTok{"transformed\_column"}\NormalTok{)}
\end{Highlighting}
\end{Shaded}

Uzorak za istraživanje čine nativni oglasi objavljivani u šestomjesečnom
razdoblju (prosinac 2021. -- svibanj 2022.) na šest najčitanijih
hrvatskih internetskih portala (index.hr, jutarnji.hr, vecernji.hr,
24sata.hr, telegram.hr, slobodnadalmacija.hr), koji su prema kriteriju
posjećenosti izdvojeni kao najutjecajniji online mediji u hrvatskom
digitalnom medijskom prostoru (Peruško, 2023), Nakon provedene strojne
predanalize koja je uključivala sve objave na hrvatskim internetskim
portalima kroz promatrano razdoblje, odnosno nešto više od 1,5 milijuna
objava na 5104 internetske stranice, izdvojen je reprezentativan uzorak
temeljen kriterija posjećenosti, tj. najveći hrvatski internetski
portali na kojima je provedena strojna pretraga po ključnim riječima i
frazama u sadržaju ili naslovu. Ključne riječi prema kojima je tako
izdvojen nativni sadržaj odnosile su se na složenice kao što su:
„plaćeni sadržaj``, „sponzorirani sadržaj``, „native``, „sadržaj
donosi``, „sadržaj nastao``, „prilog je napravljen``, „powered by``,
„tnative``, „24contenthaus``, s obzirom na to kako nativni sadržaj treba
biti istaknut kao takav u mediju. Nakon izdvajanja svih članaka kroz
šestomjesečno razdoblje koji odgovaraju ovim ključnim riječima, njih
ukupno 811, ljudskom analizom su odbačeni oni koji ne zadovoljavaju
karakteristike nativnih članaka, što se primarno odnosilo na
sponzorirane ili PR objave i druge sadržaje koji se ne odnose na nativne
oglase, a kao konačan uzorak za provedbu istraživanja su identificirana
i ukupno izdvojena 543 nativna oglasa na šest internetskih portala koji
se ističu po svojoj utjecajnosti i čitanosti u Republici Hrvatskoj.
Jednako tako, prethodno određivanju uzorka za istraživanje, utvrđeno je
kako svi internetski portali čiji su članci predmetom analize sadržaja,
uopće nude uslugu nativnog oglašavanja, kroz provjeru njihovih javno
dostupnih cjenika marketinških, oglašivačkih i usluga odnosa s javnošću.
Od ukupnog uzorka od 543 nativna oglasa, 177 objava ili 32,6 posto
objava odnosi se na internetski portal jutarnji.hr, 103 objave ili 18.97
\% na vecernji.hr, dok se 99 objava ili 18,23 \% odnosi na telegram.hr.
Na internetski portal 24sata.hr odnosi se 88 analiziranih članaka ili
16,21 \%, na index.hr 62 ili 11,42 \%, a na slobodnadalmacija.hr 14
nativnih članaka ili 2,58 posto.

Istraživanje na uzorku od 543 nativna oglasa objavljena od 1. prosinca
2021. do 30. svibnja 2022. godine na izdvojenih šest hrvatskih
internetskih portala koji zadovoljavaju kriterij posjećenosti, provedeno
je metodom analize sadržaja, pri čemu je primijenjena analitička matrica
koja je sadržavala pet općih kategorija (opći identifikacijski elementi,
sadržajno isticanje, karakteristike naslova, karakteristike sadržaja,
angažman publike), unutar kojih je izdvojeno 19 varijabli po kojima je
promatran svaki od 543 oglasa. Pri analizi su autori nazavisno čitali i
analizirali sve identificirane oglase i potom usporedili rezultate uz
potpuno poklapanje u svim varijablama među autorima.

Za potrebe ovoga istraživanja su promatrane dvije opće kategorije,
sukladno vrstama i tipologiji sadržajnog marketinga: karakteristike
naslova i karakteristike sadržaja nativnih oglasa. Varijable su se
odnosile na kategoriziranje vrsta naslova, na povezanost naslova s
brendom oglašivača te na određivanje tipa clickbait naslova, dok je
analiza samog sadržaja nativnih članaka bila usmjerena na vrstu vizualne
opreme članka (foto, video, infografika, grafika, kombinacije navedenih)
i njenu povezanost s vizualnim identitetom oglašivača, odnosno
zastupljenost različitih vrsta izvora/izjava (izvor iz
organizacije/tvrtke, brend ambasador, influencer - slavna/poznata osoba,
izravni korisnik proizvoda ili usluge, više izvora) u nativnim oglasima.

Cilj istraživanja je i utvrditi postoji li značajna korelacija između
vrsta naslova i dosegom publike, stoga su podaci o karakteristikama
naslova dovedeni u vezu s podacima o dosegu svakog pojedinog oglasa, a
istraživanje je usto i utvrdilo jesu li određene vrste naslova nativnih
oglasa zastupljenije, odnosno karakterističnije za različite industrije.
Na jednak su način uspoređene i promatrane vrste izjava koje se nalaze u
analiziranim nativnim oglasima -- s obzirom na vrstu industrije kojoj
pripada oglašivač te s obzirom na doseg te objave.

\hypertarget{results-and-discussion}{%
\section{Results and Discussion}\label{results-and-discussion}}

Od ukupnog broja od 543 analizirana nativna oglasa koja su objavljena u
šestomjesečnom razdoblju (prosinac 2021. -- svibanj 2022.) na šest
hrvatskih internetskih portala koji zadovoljavaju kriterij posjećenosti
i relevantnosti (jutarnji.hr, index.hr, vecernji.hr, 24sata.hr,
telegram.hr, slobodnadalmacija.hr), kod njih gotovo četiri petine (79,01
\%) utvrđena je zastupljenost clickbait naslova. Pri određivanju
obilježja clickbait poslužili smo se tipologijom koju su definirali
Kanižaj\ldots{} Istraživanje je pokazalo kako su najzastupljeniji oni
clickbait naslovi koje karakterizira neizvjesnost (44,38 \%). Više
kombinacija stilova u naslovima je zamijećeno kod 13,81 \% oglasa, dok
je 20,99 \% naslova bilo bez ijedne clickbait karakteristike. Naslovi u
kojima prevladava stil koji podrazumijeva korištenje brojkama pojavljuje
se u 9,39 \% slučajeva, a 5,89 \% naslova karakterizira naglašavanje
emocija. Kod 4,97 \% naslova primijećeno je korištenje nedefiniranim
zamjenicama, dok je stil obrnutog narativa registriran kod 0,55 \%
oglasa.

Tablica 1. Zastupljenost obilježja clickbait naslova kod analiziranih
nativnih oglasa (n=543)

Obilježja clickbait naslova kod nativnog oglasa Broj objava \%
Nedefinirane zamjenice 27 4,97 Neizvjesnost 241 44,38 Stil obrnutog
narativa 3 0,55 Naglašavanje emocija 32 5,89 Korištenje brojkama 51 9,39
Više kombinacija 75 13,81 Nije clickbait naslov 114 20,99

Kod velike većine promatranih nativnih oglasa (94,48 \%) nije
zabilježeno navođenje naziva brenda oglašivača u njihovu naslovu, dok je
kod 5,52 \% ili trideset članaka od njih 543, naziv brenda bio dijelom
naslova. TU ĆEMO DODATI TU RAZLIKU IZMEĐU SPONZORIRANOG ČLANKA KOJI U
PRAKSI NAVODI OGLAŠIVAČA U NASLOVU I NATIVEA KOJI TO NE PRAKTICIRA JER
JE RIJEČ O SUPTILNIJOJ OGLAŠIVAČKOJ TAKTICI BLA BLA

\begin{verbatim}
Većina naslova, njih 71,09 %, po vrsti je bilo izjavna rečenica, dok se petina naslova (20,26 posto) odnosi na upitne, a 8,95 posto na usklične rečenice. Na području karakteristika sadržaja promatranih nativnih oglasa, vrijedno je istaknuti da su baš svi analizirani oglasi imali vizualnu opremu, od čega se samostalno i u kombinacijama s drugim vrstama vizualnog sadržaja u prvome planu ističe fotografija. Kao što prikazuju podaci iz Tablice 2, u 73,67 % slučajeva, oglasi su bili opremljeni isključivo fotografskim sadržajem, dok se u 11,97 % slučajeva uz fotografiju nalazio i grafički sadržaj. Kod 11,05 % oglasa se uz fotografiju nalazio i video sadržaj, a u 2,39 % slučajeva se radilo o nativnim oglasima koji su sadržavali i foto i video i grafički sadržaj. 
\end{verbatim}

Tablica 2. Vrsta vizualne opreme analiziranih nativnih oglasa (n=543)

Vrsta vizualne opreme kod nativnih oglasa Broj objava \% Foto-sadržaj
400 73,67 Video-sadržaj 0 0 Grafika ili infografika 3 0,55 Foto i video
sadržaj 60 11,05 Foto i grafički sadržaj 65 11,97 Video i grafički
sadržaj 2 0,37 Foto, video i grafički sadržaj 13 2,39 Članak nema
vizualnu opremu 0 0

\begin{verbatim}
Logotip oglašivača, odnosno elementi vizualnog identiteta oglašivača, bili su prikazani u sklopu vizualne opreme analiziranih nativnih oglasa u 34,44 % slučajeva, dok kod 356 od 543 analizirana članka, to nije bio slučaj, odnosno, oni nisu sadržavali vizualnu poveznicu s oglašivačem. 
Od karakteristika sadržaja nativnih oglasa u ovome istraživanju, promatrane su i vrste izvora, odnosno izjava u samim člancima, što je prikazano u Tablici 3. Tako je omjer oglasa koji koriste izvore u odnosu na one koji ih ne koriste približno sličan, dok su među oglasima koji sadrže izvor najzastupljeniji oni kojima je izvor osoba iz organizacije, tj. naručitelja oglasa (19,71 %). Drugi najistaknutiji s 10,31 % zastupljenosti su oni izvori koji nisu izravno povezani s proizvodom ili uslugom, dok je u 7,55 % slučajeva prisutno više izvora, a u 6,81 % oglasa je riječ o svjedočanstvu ili izjavi osobe koja se predstavlja kao izravni korisnik proizvoda ili usluge. Kao izvori u nativnim oglasima koriste se i poznate osobe pa su tako izjave brend ambasadora korištene u 4,79 % slučajeva, a izjave influencera ili slavnih osoba su bile prisutne u 3,5 % promatranih oglasa. 
\end{verbatim}

Tablica 3. Zastupljenost različitih vrsta izvora u izjavama kod
analiziranih nativnih oglasa (n=543)

Vrste izvora / izjava kod nativnih oglasa Broj objava \% Nema izvora /
izjave 257 47,33 Izvor iz organizacije / tvrtke 107 19,71 Brend
ambasador 26 4,79 Influencer, slavna / poznata osoba 19 3,5 Izravni
korisnik proizvoda ili usluge 37 6,81 Više izvora 41 7,55 Izvor
nepovezan s proizvodom ili uslugom 56 10,31

\hypertarget{conclusion}{%
\section{Conclusion}\label{conclusion}}

Zaključak istraživanja može se sažeti kroz nekoliko ključnih točaka.
Analiza 543 nativna oglasa objavljena na šest hrvatskih internetskih
portala pokazala je značajnu prisutnost clickbait naslova, gdje je
gotovo 80 \% naslova imalo barem jednu clickbait karakteristiku, pri
čemu se najčešće koristio stil neizvjesnosti. Iako je u manjem dijelu
naslova (5,52 \%) zabilježeno navođenje brenda, većina ih je bila
neutralna po pitanju izravnog reklamiranja. Naslovi su pretežito izjavne
rečenice, dok su vizualni elementi, osobito fotografije, bili prisutni u
svim oglasima, pri čemu je logo oglašivača bio uključen u 34,44 \%
slučajeva. Na području izvora izjava, gotovo polovica članaka nije
koristila izjave, dok su u ostalim člancima najčešće izvori bile osobe
iz organizacije, s relativno malom prisutnošću influencera i slavnih
osoba. Ovi nalazi ukazuju na dominaciju clickbait naslova i značajnu
ulogu vizualne opreme u nativnom oglašavanju, dok je izravno povezivanje
s brendom i korištenje autoritativnih izvora rjeđe zastupljeno.

\renewcommand\refname{References}
\bibliography{refs.bib}


\end{document}
